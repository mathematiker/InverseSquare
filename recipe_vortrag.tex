\documentclass{beamer}
\usepackage[utf8]{inputenc}
\usepackage[ngerman]{babel} 	% Deutsche Sprachunterst\"utzung f\"ur Silbentrennung, etc.
\usepackage[T1]{fontenc}		% Erlaubt das kopieren/suchen von/nach Umlauten im pdf-Dokument
		% Encoding der tex-Dateien, erlaubt direkte Eingabe von Sonderzeichen
\usepackage{lmodern}
\usepackage{amsmath}
\usepackage{amssymb}
\usepackage{amsthm}
\usepackage{mathtools}
\mathtoolsset{showonlyrefs}
\usepackage{mathdots}


\newtheorem{thm}{Theorem}
\newtheorem{st}[thm]{Satz}
\newtheorem{lem}[thm]{Lemma}
\newtheorem{df}[thm]{Definition}
\newtheorem{kor}[thm]{Korollar}
\newtheorem{prop}[thm]{Proposition}
\newtheorem{alg}[thm]{Algorithmus}

%\theoremheadertypefont{\color{black}} % Font f\"ur Theorem-Typ (Satz, Definition, etc.)
\theoremstyle{break}

\newtheorem{nt}[thm]{Bemerkung}
\newtheorem{ex}[thm]{Beispiel}

%\usepackage{multicol}

\usepackage{multimedia}
\usepackage{graphicx}
\setbeamercovered{transparent}
\colorlet{orange}{red!70!yellow}% mes petites couleurs que j'aime bien
\colorlet{mauve}{blue!70!red}
\colorlet{brouge}{red!70!blue}
\mode<presentation>
\usetheme{Madrid}

\mode<presentation> { \setbeamercovered{transparent} }
\makeatletter
\def\beamerorig@set@color{%
  \pdfliteral{\current@color}%
  \aftergroup\reset@color
}
\def\beamerorig@reset@color{\pdfliteral{\current@color}}
\makeatother

% Delete this, if you do not want the table of contents to pop up at
% the beginning of each subsection:
\AtBeginSection[]
{
  \begin{frame}<beamer>{Outline}
    \tableofcontents[currentsection,currentsubsection]
  \end{frame}
}

\newtheorem{proposition}[theorem]{Proposition}
\newtheorem{question}[theorem]{Question}

\def\N{{\mathbb N}}
\def\Z{{\mathbb Z}}
\def\Q{{\mathbb Q}}
\def\R{{\mathbb R}}
\def\C{{\mathbb C}}
\def\T{{\mathbb T}}

% probability notations

\newcommand{\E}{{\mathbb E}}
\renewcommand{\P}{{\mathbb P}}
\newcommand{\F}{{\mathcal F}}
\newcommand{\G}{{\mathcal G}}
\renewcommand{\H}{{\mathcal H}}

% greek letters

\renewcommand{\a}{\alpha}
\renewcommand{\b}{\beta}
\newcommand{\g}{\gamma}
\renewcommand{\d}{\delta}
\newcommand{\e}{\varepsilon}
\renewcommand{\l}{\lambda}
\newcommand{\om}{\omega}
\renewcommand{\O}{\Omega}
\newcommand{\Fub}{{\rm Fub}}
\newcommand{\Ito}{{\hbox{\rm It\^o}}}

% spaces

\renewcommand{\L}{L^2(0,1)}
\newcommand{\LH}{L^2(0,1;H)}
\renewcommand{\gg}{\g(\L,E)}
\newcommand{\ggH}{\g(\LH,E)}

 \newcommand{\inv}[1]{\frac{1}{#1}}
 \newcommand{\eps}[0]{\varepsilon}
 \newcommand{\sref}[1]{(\ref{#1})}
 \newcommand{\pw}[0]{_{\textrm{PW}}}

\renewcommand{\Re}{\hbox{\rm Re}\,}
\renewcommand{\Im}{\hbox{\rm Im}\,}
\newcommand{\D}{{\mathcal D}}
\newcommand{\calL}{{\mathcal L}}
\newcommand{\n}{\Vert}
\newcommand{\one}{{{\bf 1}}}
\newcommand{\embed}{\hookrightarrow}
\newcommand{\s}{^*}
\newcommand{\lb}{\langle}
\newcommand{\rb}{\rangle}
\newcommand{\dps}{\displaystyle}
\renewcommand{\SS}{{\bf S}}
\newcommand{\limn}{\lim_{n\to\infty}}
\newcommand{\limk}{\lim_{k\to\infty}}
\newcommand{\limj}{\lim_{j\to\infty}}
\newcommand{\sumn}{\sum_{n\ge 1}}
\newcommand{\sumk}{\sum_{k\ge 1}}
\newcommand{\sumj}{\sum_{j\ge 1}}
\newcommand{\sumnN}{\sum_{n=1}^N}
\newcommand{\Ra}{\Rightarrow}
\newcommand{\LRa}{\Leftrightarrow}
\newcommand{\da}{\downarrow}
\newcommand{\wh}{\widehat}
\newcommand{\supp}{\text{\rm supp\,}}
\newcommand{\nn}{|\!|\!|}
\newcommand{\Se}{{\mathbb S}}
\newcommand{\Ae}{{\mathbb A}}

\newcommand{\umd}{\textsc{umd} }

%
\newcommand{\red}{\color{red}}
\newcommand{\green}{\color{green}}
\newcommand{\blue}{\color{blue}}
\newcommand{\black}{\color{black}}


\newcommand{\imp}[1]{#1} %\hfill$\bigstar$}
\newcommand{\journal}[1]{\textit{#1}}
\newcommand{\book}[1]{\textit{#1}}
\newcommand{\volume}[1]{\textbf{#1}}%{Vol.~#1}%
\newcommand{\name}[1]{{#1}}
\newcommand{\sign}{\rm sign\ }
\begin{document}
\title[Schrödinger-Operatoren]{Schrödinger-Operatoren mit inversquadratischem Potential} 
\author[\textsc{Hofmann}]{\textcolor{blue}{Matthias \textsc{Hofmann}.\\ \ \\
Masterseminar Funktionalanalysis
}}
 \date{Universität Stuttgart\\ Juni 2015} 
 \begin{frame}
\titlepage	
\end{frame}

\begin{frame}{Gliederung}
  \tableofcontents
  % You might wish to add the option [pausesections]
\end{frame}

\section{Einleitung}
\begin{frame}{Die Wärmegleichung}
Wir betrachten im Folgenden die W\"armeleitungsgleichung
\begin{equation}\label{heat}
\begin{cases}
\partial_t u - \Delta u =0, t\ge 0, x\in \mathbb R^N,\\
u(x,0)=u_0 \ge 0.
\end{cases}
\end{equation} \pause
Dann ist die L\"osung der W\"armeleitungsgleichung eindeutig durch
\begin{equation}
u(x,t)=\int_{\mathbb R^N} K(x-y, t) u_0(y) \, \mathrm dy,
\end{equation}
wobei $K(x)=\frac{1}{(4\pi t)^{n/2}} e^{-|x|^2/4t}$. \pause Dann ist $u(x,t)$ die klassische L\"osung des Problems \eqref{heat} 
und
\begin{equation}
T(t)\phi:=e^{\Delta t}\phi:=\int_{\mathbb R^N} K(x-y,t) \phi(y)\, \mathrm dy 
\end{equation}
definiert eine stark stetige Halbgruppe auf $L^2(\mathbb R^N)$ (vgl. \cite{engel-nagel} Beispiel II.2.12). Sie erh\"alt Positivit\"at, sodass f\"ur $\phi \ge 0$ auch $T(t) \phi \ge 0$, und ist analytisch f\"ur $t>0$.
\end{frame}

\begin{frame}
Sei $A_V= -\Delta - V(\cdot)$ ein Schr\"odinger-Operator auf $L^2(\mathbb R^N)$, sodass
\begin{equation}
0\le V\in L^\infty(\{x:|x|\ge \varepsilon\}), \quad \varepsilon >0.
\end{equation}\label{perturbed}
\pause
Wie verhalten sich L\"osungen der gest\"orten W\"armeleitungsgleichung 
%\begin{equation}
%\partial_t u +A_V u =0\quad (x\in \mathbb R^N, t\ge 0).
%\end{equation}
\begin{equation}
(K_V): \qquad \begin{cases}
\partial_t u - A_V u =0, t\ge 0, x\in \mathbb R^N,\\
u(x,0)=u_0 \ge 0.
\end{cases}
\end{equation}
\pause
F\"ur $V\in L^\infty(\mathbb R^N)$ definiert $V$ einen beschr\"ankten Multiplikationsoperator und die Existenz von klassischen L\"osungen folgt aus dem Satz \ref{bounded} (s. Appendix) f\"ur beschr\"ankte St\"orungen. %\vspace{.25cm}
Was passiert, wenn $V$ zu singul\"ar wird?
\end{frame}

\begin{frame}
Wir erlauben im Folgenden auch schwache L\"osungen von $(K_V)$. \vspace{.25cm}
\begin{df}
$u$ ist schwache L\"osung von $(K_V)$ falls, f\"ur jedes $T, R>0$, gilt
\begin{gather}\label{weak}
u\in L^1(B(0,R) \times (0,T)), Vu \in L^1(B(0,R)\times (0,T)) \text{ und }\\
\int_0^T \int_{\mathbb R^N} u (-\partial_t \phi - L\phi) \, \mathrm dx \, \mathrm dt - \int_{\mathbb R^N} f \phi(\cdot, 0) \, \mathrm dx = \int_0^T \int_{\mathbb R^N} V u \phi \, \mathrm dx \, \mathrm dt
\end{gather}
f\"ur alle $\phi \in C_c^2(\mathbb R^N\times [0,T))$.
\end{df}
\end{frame}

\begin{frame}
H. Brezis und J. L. Lions vermuteten, dass f\"ur $V(x) \le \frac{C}{|x|^{2-\varepsilon}}$ mit $C, \varepsilon>0$ keine positive L\"osungen besitzt. P. Baras und A. Goldstein l\"osten die Problemstellung.  Sei dazu $C^*(N)=( \tfrac{N-2}{2} )^2$ und $V_c(x) = \frac{c}{|x|^2}$. \vspace{.25cm}

\begin{thm}[\textsc{Baras--Goldstein} \textbf{1984}]\label{main}
Die Gleichung  $\partial_t u + A_{V_c} u =0, (x\in \mathbb R^N, t\ge0)$ besitzt positive L\"osungen (z.B. f\"ur $u(x,0)=u_0(x)$ f\"ur jedes $0\le u_0\in L^2(\mathbb R^N)$) falls $c\le C^*(N)$ und keine positive L\"osungen falls $c>C^*(N)$.
\end{thm}
%\begin{proof}[Beweisidee]
%Sei $V_n(x)=\inf\{V_c(x), n\}$ das \emph{cutoff}-Potential. Bezeichne $u_n$ die L\"osung von
%\begin{equation}\label{cutoff}
%\begin{cases}
%\partial_t u_n - \Delta u_n - V_n u_n =0, t\ge 0, x\in \mathbb R^N,\\
%u_n(x,0)=u(x,0)=f(x)\ge 0.
%\end{cases}
%\end{equation}
%Dann falls $c\le C^*(N)$ k\"onnen wir zum Grenzwert $n\to \infty$ \"ubergehen und erhalten eine L\"osung.  Die Schwierigkeit liegt im Erhalt der Konvergenz.
%\end{proof}
\end{frame}

\section{Das Cabr\'e--Martel Theorem}
\begin{frame}{Das Cabr\'e--Martel Theorem}
Wir definieren den Grundzustand von $-(\Delta +V)$ durch
\begin{equation}\label{groundstate}
\lambda_1(\Delta + V):= \inf_{\phi \in H^1(\mathbb R^N)\setminus \{0\}} \left ( \frac{\int_{\mathbb R^N} |\nabla \phi|^2\, \mathrm dx - \int_{\mathbb R^N} V\phi^2\, \mathrm dx}{\int_{\mathbb R^N} \phi^2\, \mathrm dx} \right ).
\end{equation}
\pause
\begin{thm}[\textsc{Cabr\'e--Martel} \textbf{1999}]\label{martel}
Sei $0\le V\in L^1_{\text{loc}}(\mathbb R^N), N\ge 3$.  Es folgt:
\begin{enumerate}[(i)]
\item Falls $\lambda_1(\Delta + V) > -\infty$, dann existiert eine postive L\"osung $u\in C([0,\infty), L^2(\mathbb R^N))$, sodass
\begin{equation}\label{exponential}
\|u(t)\|_{L^2(\mathbb R^N)} \le e^{\omega t} \|u_0\|_{L^2(\mathbb R^N)}, t\ge 0
\end{equation}
f\"ur ein $\omega\in \mathbb R$.
\item Falls $\lambda_1(\Delta + V)=-\infty$, dann folgt f\"ur $0\le u_0 \in L^2(\mathbb R^N)\setminus\{0\}$, dann gibt es keine positive L\"osung von $(K_V)$, sodass \eqref{exponential} erfüllt wird.
\end{enumerate}
\end{thm}
\end{frame}

\begin{frame}
\begin{equation*}
\begin{split}
\uncover<1->{\int_{\mathbb R^N} V_n \phi^2\, \mathrm dx&= \partial_t\left ( (\log u_n) \phi^2\, \mathrm dx \right )+ \int_{\mathbb R^N} \nabla u_n \cdot \nabla \left (\frac{\phi^2}{u_n} \right )\\}
\uncover<2->{&= \partial_t \left ( \int_{\mathbb R^N} (\log u_n) \phi^2\, \mathrm dx \right )+ 2 \int_{\mathbb R^N} (\nabla u_n \cdot \nabla \phi) \frac{\phi}{u_n} \, \mathrm dx \\
&\ \qquad- \int_{\mathbb R^N} |\nabla u_n |^2 \frac{\phi^2}{u_n^2} \, \mathrm dx \\}
\uncover<3->{&\le \partial_t \left ( \int_{\mathbb R^N} (\log u_n) \phi^2\, \mathrm dx \right ) + 2 \left ( \int_{\mathbb R^N} |\nabla u_n|^2 \frac{\phi^2}{u_n^2} \, \mathrm dx \right )^{1/2} \\
&\ \qquad \left ( \int_{\mathbb R^N} |\nabla \phi|^2\, \mathrm dx\right )^{1/2}- \int_{\mathbb R^N} |\nabla u_n|^2 \frac{\phi^2}{u_n^2}\, \mathrm dx \\} \uncover<4->{&\le \partial_t \left ( \int_{\mathbb R^N} (\log u_n) \phi^2\, \mathrm dx \right )+ \int_{\mathrm R^N} |\nabla \phi|^2\, \mathrm dx.}
\end{split}
\end{equation*}
\end{frame}

\begin{frame}
\begin{align*}
&\quad\int_{\mathbb R^N} V \phi^2 \, \mathrm dx - \int_{\mathbb R^N} |\nabla \phi|^2\, \mathrm dx\\
&\le \frac{1}{t-1} \left \{ \log \left [ \left (\int_{\mathbb R^N} \tilde u(t)^2 \, \mathrm dx \right )^{1/2} \left ( \int_{\mathbb R^N} \phi^4 \, \mathrm dx \right )^{1/2}\right ] - \int_{\mathbb R^N} \log(\tilde u(1))\phi^2\, \mathrm dx \right \}\\
&\le \frac{1}{2t-1} \left \{ \log  \left (\int_{\mathbb R^N} \tilde u(t)^2 \, \mathrm dx \right )+ \log \left ( \int_{\mathbb R^N} \phi^4 \, \mathrm dx \right ) - 2 \int_{\mathbb R^N} \log(\tilde u(1))\phi^2\, \mathrm dx \right \}\\
&\le \frac{1}{2t-1} \left \{ \log  \left (\int_{\mathbb R^N} u(t)^2 \, \mathrm dx \right )+ \log \left ( \int_{\mathbb R^N} \phi^4 \, \mathrm dx \right ) - 2 \int_{\mathbb R^N} \log(\tilde u(1))\phi^2\, \mathrm dx \right \}\\
&\le \frac{1}{2(t-1)}\{2\log(M) + 2\omega t + 2\log \|u_0\|_{L^2(\mathbb R^N)}  \\
&\ \qquad\qquad\qquad+ 2\log\|\phi\|_\infty- 2\int_{\mathbb R^N} \log(\tilde u(1)) \phi^2\, \mathrm dx \}
\end{align*}
\end{frame}

\section{Hardy-Ungleichung und das Baras--Goldstein-Theorem}
\begin{frame}{Hardy-Ungleichung und das Baras--Goldstein-Theorem}

Mittels Cabr\'e--Martel können wir das Baras--Goldstein-Theorem beweisen. Die Existenz von Lösungen hängt stark mit der Hardy-Ungleichung und Optimalität der Konstanten zusammen.
\begin{lem}[Hardy-Ungleichung]
Sei $N\ge 3$. Dann gilt
\begin{equation}\label{hardy}
C^*(N) \int_{\mathbb R^N} \frac{\phi^2(x)}{|x|^2}\, \mathrm dx \le \int_{\mathbb R^N} |\nabla \phi|^2\, \mathrm dx, \quad \phi \in H^1(\mathbb R^N).
\end{equation}
\end{lem}
\end{frame}

\begin{frame}
\begin{nt}
Der Cabr\'e--Martel-Ansatz wurde in \cite{goldstein-rhandi} für gestörte Kolmogorov-Gleichungen
\begin{equation}%\label{perturbed}
\begin{cases}
\partial_t u - (\Delta+V) u +\nabla \rho \cdot \nabla u=0, t\ge 0, x\in \mathbb R^N,\\
u(x,0)=u_0 \ge 0
\end{cases}
\end{equation}
studiert (eine Einführung zu Kolmogorov-Gleichungen ist z.B. in \cite{lorenzi} zu finden) und für $\rho(x)=\frac{1}{2} Bx\cdot x$, wobei $B\in \mathbb R^{N\times N}$ positive definite Matrix, beschreibt dies gerade einen \textsc{Ornstein--Uhlenbeck}-Prozess.  Für solche wurde in \cite{goldstein-rhandi} ein Analogon zum Baras--Goldstein-Theorem zur Existenz von Lösung durch den Beweis einer gewichteten Hardy-Ungleichung gezeigt.
\end{nt}
\end{frame}

\section{Appendix}
\begin{frame}[allowframebreaks]{Satz über beschränkte Störungen}
\begin{block}{Theorem}
Sei $(A, D(A))$ erzeugt von einer stark stetigen Halbgruppe $(T(t))_{t\ge 0}$ auf einem Banachraum $X$, sodass
\begin{equation}
\|T(t)\|\le M e^{\omega t}
\end{equation}
f\"ur alle $t\ge 0$ und $w\in \mathbb R$, $M\ge 1$. Falls $B\in \mathcal L(X)$, dann erzeugt $C:=A+B$ mit $D(C)=D(A)$ eine stark stetige Halbgruppe $(S(t))_{t\ge 0}$ mit
\begin{equation}
\|S(t)\|\le M e^{(w+ M\|B\|) t}
\end{equation}
f\"ur alle $t\ge 0$. 
\end{block}
\begin{block}{}
Diese erf\"ullt das \emph{\textsc{Duhamel}-Prinzip}:
\begin{equation}\label{duhamel}
S(t) x = T(t) x + \int_0^t T(t-s) B S(s) x \,\mathrm ds
\end{equation}
f\"ur alle $t\ge 0$ und $x\in X$. 
Außerdem lässt sich diese schreiben durch die stark konvergente \textsc{Dyson--Phillips}-Reihe
\begin{equation}
S(t)=\sum_{n=0}^\infty \tilde S_n(t),
\end{equation}
wobei $\tilde S_0(t):=T(t)$ und
\begin{equation}
\tilde S_{n+1} := \int_0^t T(t-s) B\tilde S_n(s)\, \mathrm dx.
\end{equation}

Insbesondere \"ubertragen sich Eigenschaften wie Postivit\"at (falls $B\ge 0$) und Analytizit\"at.
\end{block}

\end{frame}

%\begin{frame}
%\begin{lem} Seien $u_n$ Lösungen von \eqref{cutoff2}, dann gilt
%\begin{equation}
%0 <u_n \le u_{n+1}.
%\end{equation}
%\end{lem}
%
%\begin{lem} Seien $u_n$ Lösungen von \eqref{cutoff2} und positive Lösung von \eqref{perturbed}, dann gilt
%\begin{equation}
%0<u_n(t)\le u(t), \quad t\ge 0.
%\end{equation}
%\end{lem}
%\end{frame}

\begin{frame}{Trotter--Kato Approximationstheorem}
\begin{thm}[\textsc{Trotter--Neveu--Kato}]
Seien $T_k$ stark stetige, quasikontraktive Halbgruppe  auf $L^2$, so dass $0\le T_k(t) \le T_{k+1}(t)$ und
\begin{equation}
\|T_k(t)\|\le e^{\omega t},\quad \omega \in \mathbb R.
\end{equation}  
Dann existiert $T(t)f=\lim_{k\to \infty} T_k(t)f$ für alle $f\in L^2, t\ge 0$, und definiert eine stark stetige Halbgruppe $T$ auf $L^2$. 
\end{thm}
\end{frame}

\begin{thebibliography}{tief}
\bibitem{arendt} Arendt, Wolfgang; Batty, Charles J. K.; Hieber, Matthias; Neubrander, Frank:
{\it Vector-valued Laplace transforms and Cauchy problems}.
Monographs in Mathematics, 96. Birkhäuser Verlag, Basel, 2001. xii+523 pp.
\bibitem{arendt-goldstein} Arendt, Wolfgang; Goldstein, Gisèle Ruiz; Goldstein, Jerome A.: 
{\it Outgrowths of Hardy's inequality}. Recent advances in differential equations and mathematical physics, 51–68,
Contemp. Math., 412, Amer. Math. Soc., Providence, RI, 2006.
\bibitem{baras-goldstein} Baras, Pierre; Goldstein, Jerome A:
{\it The heat equation with a singular potential}.
Trans. Amer. Math. Soc. 284 (1984), no. 1, 121-139.
\bibitem{cabre-martel} Cabr\'e, Xavier; Martel, Yvan:
{\it Existence versus explosion instantan\'ee pour des \'equations de la chaleur lin\'eaires avec potentiel singulier}. (French. English, French summary) [Existence versus instantaneous blowup for linear heat equations with singular potentials]
C. R. Acad. Sci. Paris S\'er. I Math. 329 (1999), no. 11, 973-978. 
\bibitem{engel-nagel} Engel, Klaus-Jochen; Nagel, Rainer:
{\it A short course on operator semigroups}. Universitext. Springer, New York, 2006.
\bibitem{goldstein-rhandi} Goldstein, G. R.; Goldstein, J. A.; Rhandi, A.: {\it Weighted Hardy's inequality and the Kolmogorov equation perturbed by an inverse-square potential}.
Appl. Anal. 91 (2012), no. 11, 2057-2071. 
\bibitem{lady} Lady\v{z}enskaja, O. A.; Solonnikov, V. A.; Ural'ceva, N. N.
{\it Linear and quasilinear equations of parabolic type}. (Russian)
Translated from the Russian by S. Smith. Translations of Mathematical Monographs, Vol. 23 American Mathematical Society, Providence, R.I. 1968 xi+648 pp. 
\bibitem{lorenzi} Lorenzi, Luca; Bertoldi, Marcello
{\it Analytical methods for Markov semigroups}.
Pure and Applied Mathematics (Boca Raton), 283. Chapman \& Hall/CRC, Boca Raton, FL, 2007. xxxii+526 pp.
\bibitem{rhandi} Rhandi, Abdelaziz: {\it Heat and Ornstein-Uhlenbeck semigroups perturbed by an inverse-square potential}. 
                     Unpublished, 2015.
\bibitem{voigt} Voigt, Jürgen:
{\it Absorption semigroups, their generators, and Schrödinger semigroups}.
J. Funct. Anal. 67 (1986), no. 2, 167–205. 
\end{thebibliography} 



\end{document} 
