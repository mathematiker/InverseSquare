\documentclass[11pt]{article}
\usepackage[ngerman]{babel} 	% Deutsche Sprachunterstützung für Silbentrennung, etc.
\usepackage[T1]{fontenc}		% Erlaubt das kopieren/suchen von/nach Umlauten im pdf-Dokument
\usepackage[utf8]{inputenc}		% Encoding der tex-Dateien, erlaubt direkte Eingabe von Sonderzeichen
\usepackage{lmodern}			% Schönere Schriftart für pdf-Dokumente

\usepackage{enumerate}			% Aufzählungen wie a), (I), etc. als optionales Argument
\usepackage{parskip}			% Entfernt Einrückung bei Absätzen und fügt vertikalen Abstand ein
\usepackage{float}              % Erlaubt [H] Option für table/figure
%\usepackage{scrhack} 			% Wegen float-package, das eine warnung mit scrbook erzeugt

\usepackage{algpseudocode}		% algorithmicx package für Pseudo-Code
\usepackage{verbatim}			% Für Ascii-Art ;-)

\usepackage{amsmath}
\usepackage{amssymb}
\usepackage{amsthm}
\usepackage{mathtools}
\mathtoolsset{showonlyrefs}
\usepackage{mathdots}

\usepackage{fancyhdr} % Required for custom headers
\usepackage{lastpage} % Required to determine the last page for the footer
\usepackage{extramarks} % Required for headers and footers
\usepackage{graphicx} % Required to insert images
%\usepackage{lipsum} % Used for inserting dummy 'Lorem ipsum' text into the template
\usepackage{amsmath}
%\usepackage{amsfont}
%\usepackage{amssymb}

\newtheorem{thm}{Theorem}
\newtheorem{st}[thm]{Satz}
\newtheorem{lem}[thm]{Lemma}
\newtheorem{df}[thm]{Definition}
\newtheorem{kor}[thm]{Korollar}
\newtheorem{prop}[thm]{Proposition}
\newtheorem{alg}[thm]{Algorithmus}

%\theoremheadertypefont{\color{black}} % Font für Theorem-Typ (Satz, Definition, etc.)
\theoremstyle{break}

\newtheorem{nt}[thm]{Bemerkung}
\newtheorem{ex}[thm]{Beispiel}

\newtheorem*{note}{Bemerkung}

\usepackage{multicol}
% Margins
\topmargin=-0.5in
\evensidemargin=0in
\oddsidemargin=-0.5in
\textwidth=7.5in
\textheight=9.0in
\headsep=0.5in 

\pagestyle{fancy}

\rhead{Matthias \textsc{Hofmann}} % Top right header
\lhead{Schrödinger mit inversquadratischem Potential\\
\today}
\chead{ }
%\title{}

\begin{document}
\begin{titlepage}

\begin{center}


% Oberer Teil der Titelseite:

\textsc{\LARGE Universität Stuttgart}\\[1.5cm]

\textsc{\Large Masterseminar Funktionalanalysis}\\[0.5cm]


% Title
\newcommand{\HRule}{\rule{\linewidth}{0.5mm}}
\HRule \\[0.4cm]
{ \huge \bfseries Schrödingeroperatoren mit inversquadratischem Potential}\\[0.4cm]

\HRule \\[1.5cm]

% Author and supervisor

\begin{center} \Large
Matthias \textsc{Hofmann}
\end{center}

\hfill

\vfill

% Unterer Teil der Seite
{\large \today}

\end{center}

\end{titlepage}
\section{Einführung}
Wir betrachten im Folgenden die Wärmeleitungsgleichung
\begin{equation}\label{heat}
\begin{cases}
\partial_t u - \Delta u =0, t\ge 0, x\in \mathbb R^N,\\
u(x,0)=u_0 \ge 0.
\end{cases}
\end{equation}
Dann ist die Lösung der Wärmeleitungsgleichung eindeutig durch
\begin{equation}
u(x,t)=\int_{\mathbb R^N} K(x-y, t) u_0(y) \, \mathrm dy,
\end{equation}
wobei $K(x)=\frac{1}{(4\pi t)^{n/2}} e^{-|x|^2/4t}$. Dann ist $u(x,t)$ die klassische Lösung des Problems \eqref{heat} 
und
\begin{equation}
T(t)\phi:=e^{\Delta t}\phi:=\int_{\mathbb R^N} K(x-y,t) \phi(y)\, \mathrm dy 
\end{equation}
definiert eine stark stetige Halbgruppe auf $L^2(\mathbb R^N)$ (vgl. \cite{engel-nagel} Beispiel II.2.12). Sie erhält Positivität, sodass für $\phi \ge 0$ auch $T(t) \phi \ge 0$, und ist analytisch für $t>0$.

Sei $A_V= -\Delta - V(\cdot)$ ein Schrödinger-Operator auf $L^2(\mathbb R^N)$, sodass
\begin{equation}
0\le V\in L^\infty(\{x:|x|\ge \varepsilon\})
\end{equation}\label{perturbed}
für jedes $\varepsilon>0$. Wie verhalten sich Lösungen der gestörten Wärmeleitungsgleichung 
\begin{equation}
\partial_t u +A_V u =0\quad (x\in \mathbb R^N, t\ge 0).
\end{equation}
Für $V\in L^\infty(\mathbb R^N)$ definiert $V$ einen beschränkten Multiplikationsoperator und die Existenz von klassischen Lösungen folgt aus dem Satz \ref{bounded} (s. Appendix) für beschränkte Störungen. %\vspace{.25cm}
Was passiert, wenn $V$ zu singulär wird?

Wir erlauben im Folgenden auch schwache Lösungen von \eqref{heat}. \vspace{.25cm}
\begin{df}
$u$ ist schwache Lösung von \eqref{perturbed} falls, für jedes $T, R>0$, gilt
\begin{gather}
u\in L^1(B(0,R) \times (0,T)), Vu \in L^1(B(0,R)\times (0,T)) \text{ und }\\
\int_0^T \int_{\mathbb R^N} u (-\partial_t \phi - L\phi) \, \mathrm dx \, \mathrm dt - \int_{\mathbb R^N} f \phi(\cdot, 0) \, \mathrm dx = \int_0^T \int_{\mathbb R^N} V u \phi \, \mathrm dx \, \mathrm dt
\end{gather}
für alle $\phi \in C_c^2(\mathbb R^N\times [0,T))$.
\end{df}

H. Brezis und J. L. Lions vermuteten, dass für $V(x) \le \frac{C}{|x|^{2-\varepsilon}}$ mit $C, \varepsilon>0$ keine positive Lösungen besitzt. P. Baras und A. Goldstein lösten die Problemstellung.  Sei dazu $C^*(N)=( \tfrac{N-2}{N} )^2$ und $V_c(x) = \frac{c}{|x|^2}$. \vspace{.25cm}

\begin{thm}[Baras--Goldstein]
Die Gleichung  $\partial_t u + A_{V_c} u =0, (x\in \mathbb R^N, t\ge0)$ besitzt positive Lösungen (z.B. für $u(x,0)=u_0(x)$ für jedes $0\le u_0\in L^2(\mathbb R^N)$) falls $c\le C^*(N)$ und keine positive Lösungen falls $c>C^*(N)$.
\end{thm}
\begin{proof}[Beweisidee]
Sei $V_n(x)=\inf\{V_c(x), n\}$ 
\end{proof}

\appendix
\section{Appendix}
\subsection{Halbgruppen bei Beschränkten Störungen}
\begin{thm}\label{bounded}
Sei $(A, D(A))$ erzeugt von einer stark stetigen Halbgruppe $(T(t))_{t\ge 0}$ auf einem Banachraum $X$, sodass
\begin{equation}
\|T(t)\|\le M e^{\omega t}
\end{equation}
für alle $t\ge 0$ und $w\in \mathbb R$, $M\ge 1$. Falls $B\in \mathcal L(X)$, dann erzeugt $C:=A+B$ mit $D(C)=D(A)$ eine stark stetige Halbgruppe mit
\begin{equation}
\|S(t)\|\le M e^{(w+ M\|B\|) t}
\end{equation}
für alle $t\ge 0$. Diese erfüllt das \emph{Duhamel-Prinzip}:
\begin{equation}
S(t) x = T(t) x + \int_0^t T(t-s) B S(s) x \,\mathrm ds
\end{equation}
für alle $t\ge 0$ und $x\in X$. Insbesondere übertragen sich Eigenschaften wie Postivität (falls $B\ge 0$) und Analytizität.
\end{thm}
\begin{proof}
Wir verweisen auf \cite{engel-nagel} Satz III.1.3 und Korollar III.1.7. Die Positivität der Halbgruppe folgt aus \cite{engel-nagel} Korollar VI.2.4 und die Analytizät nach Theorem III.2.10.
\end{proof}

\begin{thebibliography}{tief}
\bibitem{baras-goldstein} Baras, Pierre; Goldstein, Jerome A:
{\it The heat equation with a singular potential}.
Trans. Amer. Math. Soc. 284 (1984), no. 1, 121–139.
\bibitem{cabre-martel} Cabré, Xavier; Martel, Yvan:
{\it Existence versus explosion instantanée pour des équations de la chaleur linéaires avec potentiel singulier}. (French. English, French summary) [Existence versus instantaneous blowup for linear heat equations with singular potentials]
C. R. Acad. Sci. Paris Sér. I Math. 329 (1999), no. 11, 973–978. 
\bibitem{engel-nagel} Engel, Klaus-Jochen; Nagel, Rainer:
{\it A short course on operator semigroups}. Universitext. Springer, New York, 2006.
\bibitem{goldstein-rhandi} Goldstein, G. R.; Goldstein, J. A.; Rhandi, A.: {\it Weighted Hardy's inequality and the Kolmogorov equation perturbed by an inverse-square potential}.
Appl. Anal. 91 (2012), no. 11, 2057–2071. 
\bibitem{rhandi} Rhandi, Abdelaziz: {\it Heat and Ornstein-Uhlenbeck semigroups perturbed by an inverse-square potential}. 
                     Unpublished, 2015.
\end{thebibliography} 








































\end{document}