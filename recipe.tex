\documentclass[11pt]{article}
\usepackage[utf8]{inputenc}
\usepackage[ngerman]{babel} 	% Deutsche Sprachunterst\"utzung f\"ur Silbentrennung, etc.
\usepackage[T1]{fontenc}		% Erlaubt das kopieren/suchen von/nach Umlauten im pdf-Dokument
		% Encoding der tex-Dateien, erlaubt direkte Eingabe von Sonderzeichen
\usepackage{lmodern}			% Sch\"onere Schriftart f\"ur pdf-Dokumente

\usepackage{enumerate}			% Aufz\"ahlungen wie a), (I), etc. als optionales Argument
\usepackage{parskip}			% Entfernt Einr\"uckung bei Abs\"atzen und f\"ugt vertikalen Abstand ein
\usepackage{float}              % Erlaubt [H] Option f\"ur table/figure
%\usepackage{scrhack} 			% Wegen float-package, das eine warnung mit scrbook erzeugt

\usepackage{algpseudocode}		% algorithmicx package f\"ur Pseudo-Code
\usepackage{verbatim}			% F\"ur Ascii-Art ;-)

\usepackage{amsmath}
\usepackage{amssymb}
\usepackage{amsthm}
\usepackage{mathtools}
\mathtoolsset{showonlyrefs}
\usepackage{mathdots}

\usepackage{fancyhdr} % Required for custom headers
\usepackage{lastpage} % Required to determine the last page for the footer
\usepackage{extramarks} % Required for headers and footers
\usepackage{graphicx} % Required to insert images
%\usepackage{lipsum} % Used for inserting dummy 'Lorem ipsum' text into the template
\usepackage{amsmath}
%\usepackage{amsfont}
%\usepackage{amssymb}

\newtheorem{thm}{Theorem}
\newtheorem{st}[thm]{Satz}
\newtheorem{lem}[thm]{Lemma}
\newtheorem{df}[thm]{Definition}
\newtheorem{kor}[thm]{Korollar}
\newtheorem{prop}[thm]{Proposition}
\newtheorem{alg}[thm]{Algorithmus}

%\theoremheadertypefont{\color{black}} % Font f\"ur Theorem-Typ (Satz, Definition, etc.)
\theoremstyle{break}

\newtheorem{nt}[thm]{Bemerkung}
\newtheorem{ex}[thm]{Beispiel}

\newtheorem*{note}{Bemerkung}

\usepackage{multicol}
% Margins
\topmargin=-0.5in
\evensidemargin=0in
\oddsidemargin=-0.5in
\textwidth=7.5in
\textheight=9.0in
\headsep=0.5in 

\pagestyle{fancy}

\rhead{Matthias \textsc{Hofmann}} % Top right header
\lhead{Schr\"odinger mit inversquadratischem Potential\\
\today}
\chead{ }
%\title{}

\begin{document}
\begin{titlepage}

\begin{center}


% Oberer Teil der Titelseite:

\textsc{\LARGE Universit\"at Stuttgart}\\[1.5cm]

\textsc{\Large Masterseminar Funktionalanalysis}\\[0.5cm]


% Title
\newcommand{\HRule}{\rule{\linewidth}{0.5mm}}
\HRule \\[0.4cm]
{ \huge \bfseries Schr\"odingeroperatoren mit inversquadratischem Potential}\\[0.4cm]

\HRule \\[1.5cm]

% Author and supervisor

\begin{center} \Large
Matthias \textsc{Hofmann}
\end{center}

\hfill

\vfill

% Unterer Teil der Seite
{\large \today}

\end{center}

\end{titlepage}
\section{Einf\"uhrung}
Wir betrachten im Folgenden die W\"armeleitungsgleichung
\begin{equation}\label{heat}
\begin{cases}
\partial_t u - \Delta u =0, t\ge 0, x\in \mathbb R^N,\\
u(x,0)=u_0 \ge 0.
\end{cases}
\end{equation}
Dann ist die L\"osung der W\"armeleitungsgleichung eindeutig durch
\begin{equation}
u(x,t)=\int_{\mathbb R^N} K(x-y, t) u_0(y) \, \mathrm dy,
\end{equation}
wobei $K(x)=\frac{1}{(4\pi t)^{n/2}} e^{-|x|^2/4t}$. Dann ist $u(x,t)$ die klassische L\"osung des Problems \eqref{heat} 
und
\begin{equation}
T(t)\phi:=e^{\Delta t}\phi:=\int_{\mathbb R^N} K(x-y,t) \phi(y)\, \mathrm dy 
\end{equation}
definiert eine stark stetige Halbgruppe auf $L^2(\mathbb R^N)$ (vgl. \cite{engel-nagel} Beispiel II.2.12). Sie erh\"alt Positivit\"at, sodass f\"ur $\phi \ge 0$ auch $T(t) \phi \ge 0$, und ist analytisch f\"ur $t>0$.

Sei $A_V= -\Delta - V(\cdot)$ ein Schr\"odinger-Operator auf $L^2(\mathbb R^N)$, sodass
\begin{equation}
0\le V\in L^\infty(\{x:|x|\ge \varepsilon\})
\end{equation}\label{perturbed}
f\"ur jedes $\varepsilon>0$. Wie verhalten sich L\"osungen der gest\"orten W\"armeleitungsgleichung 
%\begin{equation}
%\partial_t u +A_V u =0\quad (x\in \mathbb R^N, t\ge 0).
%\end{equation}
\begin{equation}\label{perturbed}
\begin{cases}
\partial_t u - A_V u =0, t\ge 0, x\in \mathbb R^N,\\
u(x,0)=u_0 \ge 0.
\end{cases}
\end{equation}
F\"ur $V\in L^\infty(\mathbb R^N)$ definiert $V$ einen beschr\"ankten Multiplikationsoperator und die Existenz von klassischen L\"osungen folgt aus dem Satz \ref{bounded} (s. Appendix) f\"ur beschr\"ankte St\"orungen. %\vspace{.25cm}
Was passiert, wenn $V$ zu singul\"ar wird?

Wir erlauben im Folgenden auch schwache L\"osungen von \eqref{perturbed}. \vspace{.25cm}
\begin{df}
$u$ ist schwache L\"osung von \eqref{perturbed} falls, f\"ur jedes $T, R>0$, gilt
\begin{gather}\label{weak}
u\in L^1(B(0,R) \times (0,T)), Vu \in L^1(B(0,R)\times (0,T)) \text{ und }\\
\int_0^T \int_{\mathbb R^N} u (-\partial_t \phi - L\phi) \, \mathrm dx \, \mathrm dt - \int_{\mathbb R^N} f \phi(\cdot, 0) \, \mathrm dx = \int_0^T \int_{\mathbb R^N} V u \phi \, \mathrm dx \, \mathrm dt
\end{gather}
f\"ur alle $\phi \in C_c^2(\mathbb R^N\times [0,T))$.
\end{df}

H. Brezis und J. L. Lions vermuteten, dass f\"ur $V(x) \le \frac{C}{|x|^{2-\varepsilon}}$ mit $C, \varepsilon>0$ keine positive L\"osungen besitzt. P. Baras und A. Goldstein l\"osten die Problemstellung.  Sei dazu $C^*(N)=( \tfrac{N-2}{N} )^2$ und $V_c(x) = \frac{c}{|x|^2}$. \vspace{.25cm}

\begin{thm}[\textsc{Baras--Goldstein} \textbf{1984}]\label{main}
Die Gleichung  $\partial_t u + A_{V_c} u =0, (x\in \mathbb R^N, t\ge0)$ besitzt positive L\"osungen (z.B. f\"ur $u(x,0)=u_0(x)$ f\"ur jedes $0\le u_0\in L^2(\mathbb R^N)$) falls $c\le C^*(N)$ und keine positive L\"osungen falls $c>C^*(N)$.
\end{thm}
\begin{proof}[Beweisidee]
Sei $V_n(x)=\inf\{V_c(x), n\}$ das \emph{cutoff}-Potential. Bezeichne $u_n$ die L\"osung von
\begin{equation}\label{cutoff}
\begin{cases}
\partial_t u_n - \Delta u_n - V_n u_n =0, t\ge 0, x\in \mathbb R^N,\\
u_n(x,0)=u(x,0)=f(x)\ge 0.
\end{cases}
\end{equation}
Dann falls $c\le C^*(N)$ k\"onnen wir zum Grenzwert $n\to \infty$ \"ubergehen und erhalten eine L\"osung.  Die Schwierigkeit liegt im Erhalt der Konvergenz.
\end{proof}

\section{Das Cabr\'e--Martel Theorem}

Wir definieren den Grundzustand von $-(\Delta +V)$ durch
\begin{equation}\label{groundstate}
\lambda_1(\Delta + V):= \inf_{\phi \in H^1(\mathbb R^N)\setminus \{0\}} \left ( \frac{\int_{\mathbb R^N} |\nabla \phi|^2\, \mathrm dx - \int_{\mathbb R^N} V\phi^2\, \mathrm dx}{\int_{\mathbb R^N} \phi^2\, \mathrm dx} \right ).
\end{equation}
X. Cabr\'e und Y. Martel zeigten den Zusammenhang zwischen Existenz von schwachen L\"osungen von \eqref{heat} und der Existenz eines Grundzustands. 
\begin{thm}[\textsc{Cabr\'e--Martel} \textbf{1999}]\label{martel}
Sei $0\le V\in L^1_{\text{loc}}(\mathbb R^N), N\ge 3$.  Es folgt:
\begin{enumerate}[(i)]
\item Falls $\lambda_1(\Delta + V) > -\infty$, dann existiert eine postive L\"osung $u\in C([0,\infty), L^2(\mathbb R^N))$, sodass
\begin{equation}\label{exponential}
\|u(t)\|_{L^2(\mathbb R^N)} \le e^{\omega t} \|u_0\|_{L^2(\mathbb R^N)}, t\ge 0
\end{equation}
f\"ur ein $\omega\in \mathbb R$.
\item Falls $\lambda_1(\Delta + V)=-\infty$, dann folgt f\"ur $0\le u_0 \in L^2(\mathbb R^N)\setminus\{0\}$, dann gibt es keine positive L\"osung von \eqref{heat}, die \eqref{exponential}.
\end{enumerate}
\end{thm}
\begin{proof}
%\begin{enumerate}[(i)]
(i) Betrachte im Folgenden die L\"osungen $u_n$ von
\begin{equation}\label{cutoff2}
\begin{cases}
\partial_t u_n - \Delta u_n - V_n u_n =0, t\ge 0, x\in \mathbb R^N,\\
u_n(x,0)=u(x,0)=f(x)\ge 0.
\end{cases}
\end{equation}
Nach Theorem \ref{bounded} erzeugt $\Delta + V_n$ eine positive, analytische, stark stetige Halbgruppe $S_n(\cdot)$ auf $L^2(\mathbb R^N)$. Es folgt mit \eqref{duhamel}
\begin{equation}
0 \le u_n \le u_{n+1}, \quad n=1,2,3,\ldots
\end{equation}
Multiplizieren wir \eqref{cutoff2} mit $u_n$ und integrieren, dann folgt
\begin{equation}
\frac{1}{2} \int_{\mathbb R^N} \partial_t (u_n)^2\, \mathrm dx \le - \int_{\mathbb R^N} |\nabla u_n|^2 + \int_{\mathbb R^N} V u_n^2\, \mathrm d\mu.
\end{equation}
So erhalten wir mit der Definition von \eqref{groundstate}
\begin{equation}
\frac{1}{2} \int_{\mathbb R^N} \partial_t (u_n)^2\, \mathrm dx \le -\lambda_1(\Delta+V) \int_{\mathbb R^n} u_n^2 dx.
\end{equation}
Und somit
\begin{equation}
\|u_n(t)\|_{L^2(\mathbb R^N)} \le e^{-\lambda_1(\Delta +V) t} \|u_0\|_{L^2(\mathbb R^N)}, t\ge 0.
\end{equation}
Daher folgt die lokal gleichm\"a{\ss}ige Beschr\"anktheit der zugeh\"origen Halbgruppen mit
\begin{equation}
\|S_n(t)\| \le e^{-\lambda_1(\Delta + V)}
\end{equation}
Nach dem Trotter-Neveu-Kato Theorem (vgl. Appendix) folgt die Existenz einer stark stetigen Halbgruppe $S(t)$, sodass $S_n(t) \to S(t), t\ge 0$ im staken Sinne konvergiert.  Dann folgt im Grenzwert f\"ur $u(t) = S(t) u_0\in C([0,T],\mathbb L^2(\mathbb R^N))$, dass dies eine schwache L\"osung zu \eqref{perturbed} ist.  

(ii) Angenommen es g\"abe eine postive schwache L\"osung zu \eqref{perturbed} mit Anfangswert $u_0\ge 0$.  Sei $u_n$ die eindeutige postive L\"osung zu \eqref{cutoff2}. Dann gilt das Maximumsprinzip (vgl. Appendix)
\begin{equation}\label{maximum}
0<u_n \le u.
\end{equation}
Mit monotoner Konvergenz folgt, die Existenz eines punktweisen Grenzwerts $u_n(t)\to \tilde u(t), t\ge 0$, die sogenannte milde L\"osung des Problems.  

Multiplizieren wir \eqref{cutoff2} mit $\frac{\phi^2}{u_n}$ und integrieren wir, so ergibt sich
\begin{equation}
\int_{\mathbb R^N} V_n \phi^2\, \mathrm dx \le \partial_t \left ( \int_{\mathbb R^N} (\log u_n) \phi^2\, \mathrm dx \right )+ \int_{\mathrm R^N} |\nabla \phi|^2\, \mathrm dx.
\end{equation}
Integrieren wir f\"ur $t\in (1,\infty)$, so folgt
\begin{equation}
(t-1) \int_{\mathbb R^N} V_n \phi^2\, \mathrm dx \le \int_{\mathbb R^N} \log \left ( \frac{u_n}{u_0} \right )\phi^2\, \mathrm dx + (t-1) \int_{\mathbb R^N}|\nabla \phi|^2\, \mathrm dx
\end{equation}
f�r jedes $t>1$. Lassn wir $n\to \infty$ folgt mit Lebesgue
\begin{equation}
\int_{\mathbb R^N} V\phi^2 \, \mathrm d\mu - \int_{\mathbb R^N} |\nabla \phi|^2 \, \mathrm dx \le \frac{1}{t-1} \left [ \int_{\mathbb R^N} \log(\tilde u(t)) \phi^2 \, \mathrm dx -\int_{\mathbb R^N} \log(\tilde u(1))\phi^2\, \mathrm dx \right ]
\end{equation}
f�r jedes $t>1$. Mit Jensenscher und H�lderscher Ungleichungen folgt
\begin{equation}
\int_{\mathbb R^N} V\phi^2 \, \mathrm dx - \int_{\mathbb R^N} |\nabla \phi|^2\, \mathrm dx \le \frac{1}{2(t-1)} \left \{ 2\log(M) + 2\omega t + 2\log\|u_0\|_{L^2(\mathbb R^N)} + 2 \log \|\phi\|_\infty - 2\int_{\mathbb R^N} \log(\tilde u(1))\phi^2\, \mathrm dx \right \}.
\end{equation}
Im Grenzwert $t\to \infty$, erhalten wir
\begin{equation}
\int_{\mathbb R^N} V\phi^2\, \mathrm dx - \int_{\mathbb R^N} |\nabla \phi|^2\, \mathrm dx \le \omega \int_{\mathbb R^N} \phi^2\, \mathrm dx.
\end{equation}
Mit Dichtheit von $C_c^\infty(\mathbb R^N)$ in $H_0^1(\mathbb R^N)$ folgt $\lambda_1(\Delta +V) >-\infty$. Ein Widerspruch zur Voraussetzung.
%\end{enumerate}
\end{proof}
\begin{nt}\label{bem}
Die Bedingung \eqref{exponential} in der Nichtexistenz ist tats�chlich nicht notwendig. Dies h�ngt mit dem Maximumsprinzip \eqref{maximum} zusammen. In \cite{baras-goldstein} wurde mittels eines Blowup-Arguments gezeigt, dass die L�sungen $u_n(x,t)$ f�r $t>0$ �berall unbeschr�nkt sind.
\end{nt}
\section{Hardy-Ungleichung und das Baras--Goldstein-Theorem}
Mittels Theorem \ref{martel} in Verbindung mit Bemerkung \ref{bem} k�nnen wir das Theorem  \ref{main} beweisen. Tats�chlich h�ngt die Existenz von L�sungen stark mit der Hardy-Ungleichung zusammen.    
\appendix
\section{Appendix}
\subsection{Halbgruppen bei Beschr\"ankten St\"orungen}
\begin{thm}\label{bounded}
Sei $(A, D(A))$ erzeugt von einer stark stetigen Halbgruppe $(T(t))_{t\ge 0}$ auf einem Banachraum $X$, sodass
\begin{equation}
\|T(t)\|\le M e^{\omega t}
\end{equation}
f\"ur alle $t\ge 0$ und $w\in \mathbb R$, $M\ge 1$. Falls $B\in \mathcal L(X)$, dann erzeugt $C:=A+B$ mit $D(C)=D(A)$ eine stark stetige Halbgruppe mit
\begin{equation}
\|S(t)\|\le M e^{(w+ M\|B\|) t}
\end{equation}
f\"ur alle $t\ge 0$. Diese erf\"ullt das \emph{Duhamel-Prinzip}:
\begin{equation}\label{duhamel}
S(t) x = T(t) x + \int_0^t T(t-s) B S(s) x \,\mathrm ds
\end{equation}
f\"ur alle $t\ge 0$ und $x\in X$. Insbesondere \"ubertragen sich Eigenschaften wie Postivit\"at (falls $B\ge 0$) und Analytizit\"at.
\end{thm}
\begin{proof}
Wir verweisen auf \cite{engel-nagel} Satz III.1.3 und Korollar III.1.7. Die Positivit\"at der Halbgruppe folgt aus \cite{engel-nagel} Korollar VI.2.4 und die Analytiz\"at nach Theorem III.2.10. 
\end{proof}

\begin{thebibliography}{tief}
\bibitem{baras-goldstein} Baras, Pierre; Goldstein, Jerome A:
{\it The heat equation with a singular potential}.
Trans. Amer. Math. Soc. 284 (1984), no. 1, 121-139.
\bibitem{cabre-martel} Cabr\'e, Xavier; Martel, Yvan:
{\it Existence versus explosion instantan\'ee pour des \'equations de la chaleur lin\'eaires avec potentiel singulier}. (French. English, French summary) [Existence versus instantaneous blowup for linear heat equations with singular potentials]
C. R. Acad. Sci. Paris S\'er. I Math. 329 (1999), no. 11, 973-978. 
\bibitem{engel-nagel} Engel, Klaus-Jochen; Nagel, Rainer:
{\it A short course on operator semigroups}. Universitext. Springer, New York, 2006.
\bibitem{goldstein-rhandi} Goldstein, G. R.; Goldstein, J. A.; Rhandi, A.: {\it Weighted Hardy's inequality and the Kolmogorov equation perturbed by an inverse-square potential}.
Appl. Anal. 91 (2012), no. 11, 2057-2071. 
\bibitem{rhandi} Rhandi, Abdelaziz: {\it Heat and Ornstein-Uhlenbeck semigroups perturbed by an inverse-square potential}. 
                     Unpublished, 2015.
\end{thebibliography} 








































\end{document}
package{mathtools}
\mathtoolsset{showonlyrefs}
\usepackage{mathdots}

\usepackage{fancyhdr} % Required for custom headers
\usepackage{lastpage} % Required to determine the last page for the footer
\usepackage{extramarks} % Required for headers and footers
\usepackage{graphicx} % Required to insert images
%\usepackage{lipsum} % Used for inserting dummy 'Lorem ipsum' text into the template
\usepackage{amsmath}
%\usepackage{amsfont}
%\usepackage{amssymb}

\newtheorem{thm}{Theorem}
\newtheorem{st}[thm]{Satz}
\newtheorem{lem}[thm]{Lemma}
\newtheorem{df}[thm]{Definition}
\newtheorem{kor}[thm]{Korollar}
\newtheorem{prop}[thm]{Proposition}
\newtheorem{alg}[thm]{Algorithmus}

%\theoremheadertypefont{\color{black}} % Font f\"ur Theorem-Typ (Satz, Definition, etc.)
\theoremstyle{break}

\newtheorem{nt}[thm]{Bemerkung}
\newtheorem{ex}[thm]{Beispiel}

\newtheorem*{note}{Bemerkung}

\usepackage{multicol}
% Margins
\topmargin=-0.5in
\evensidemargin=0in
\oddsidemargin=-0.5in
\textwidth=7.5in
\textheight=9.0in
\headsep=0.5in 

\pagestyle{fancy}

\rhead{Matthias \textsc{Hofmann}} % Top right header
\lhead{Schr\"odinger mit inversquadratischem Potential\\
\today}
\chead{ }
%\title{}

\begin{document}
\begin{titlepage}

\begin{center}


% Oberer Teil der Titelseite:

\textsc{\LARGE Universit\"at Stuttgart}\\[1.5cm]

\textsc{\Large Masterseminar Funktionalanalysis}\\[0.5cm]


% Title
\newcommand{\HRule}{\rule{\linewidth}{0.5mm}}
\HRule \\[0.4cm]
{ \huge \bfseries Schr\"odingeroperatoren mit inversquadratischem Potential}\\[0.4cm]

\HRule \\[1.5cm]

% Author and supervisor

\begin{center} \Large
Matthias \textsc{Hofmann}
\end{center}

\hfill

\vfill

% Unterer Teil der Seite
{\large \today}

\end{center}

\end{titlepage}
\section{Einf\"uhrung}
Wir betrachten im Folgenden die W\"armeleitungsgleichung
\begin{equation}\label{heat}
\begin{cases}
\partial_t u - \Delta u =0, t\ge 0, x\in \mathbb R^N,\\
u(x,0)=u_0 \ge 0.
\end{cases}
\end{equation}
Dann ist die L\"osung der W\"armeleitungsgleichung eindeutig durch
\begin{equation}
u(x,t)=\int_{\mathbb R^N} K(x-y, t) u_0(y) \, \mathrm dy,
\end{equation}
wobei $K(x)=\frac{1}{(4\pi t)^{n/2}} e^{-|x|^2/4t}$. Dann ist $u(x,t)$ die klassische L\"osung des Problems \eqref{heat} 
und
\begin{equation}
T(t)\phi:=e^{\Delta t}\phi:=\int_{\mathbb R^N} K(x-y,t) \phi(y)\, \mathrm dy 
\end{equation}
definiert eine stark stetige Halbgruppe auf $L^2(\mathbb R^N)$ (vgl. \cite{engel-nagel} Beispiel II.2.12). Sie erh\"alt Positivit\"at, sodass f\"ur $\phi \ge 0$ auch $T(t) \phi \ge 0$, und ist analytisch f\"ur $t>0$.

Sei $A_V= -\Delta - V(\cdot)$ ein Schr\"odinger-Operator auf $L^2(\mathbb R^N)$, sodass
\begin{equation}
0\le V\in L^\infty(\{x:|x|\ge \varepsilon\})
\end{equation}\label{perturbed}
f\"ur jedes $\varepsilon>0$. Wie verhalten sich L\"osungen der gest\"orten W\"armeleitungsgleichung 
\begin{equation}
\partial_t u +A_V u =0\quad (x\in \mathbb R^N, t\ge 0).
\end{equation}
F\"ur $V\in L^\infty(\mathbb R^N)$ definiert $V$ einen beschr\"ankten Multiplikationsoperator und die Existenz von klassischen L\"osungen folgt aus dem Satz \ref{bounded} (s. Appendix) f\"ur beschr\"ankte St\"orungen. %\vspace{.25cm}
Was passiert, wenn $V$ zu singul\"ar wird?

Wir erlauben im Folgenden auch schwache L\"osungen von \eqref{heat}. \vspace{.25cm}
\begin{df}
$u$ ist schwache L\"osung von \eqref{perturbed} falls, f\"ur jedes $T, R>0$, gilt
\begin{gather}
u\in L^1(B(0,R) \times (0,T)), Vu \in L^1(B(0,R)\times (0,T)) \text{ und }\\
\int_0^T \int_{\mathbb R^N} u (-\partial_t \phi - L\phi) \, \mathrm dx \, \mathrm dt - \int_{\mathbb R^N} f \phi(\cdot, 0) \, \mathrm dx = \int_0^T \int_{\mathbb R^N} V u \phi \, \mathrm dx \, \mathrm dt
\end{gather}
f\"ur alle $\phi \in C_c^2(\mathbb R^N\times [0,T))$.
\end{df}

H. Brezis und J. L. Lions vermuteten, dass f\"ur $V(x) \le \frac{C}{|x|^{2-\varepsilon}}$ mit $C, \varepsilon>0$ keine positive L\"osungen besitzt. P. Baras und A. Goldstein l\"osten die Problemstellung.  Sei dazu $C^*(N)=( \tfrac{N-2}{N} )^2$ und $V_c(x) = \frac{c}{|x|^2}$. \vspace{.25cm}

\begin{thm}[Baras--Goldstein]
Die Gleichung  $\partial_t u + A_{V_c} u =0, (x\in \mathbb R^N, t\ge0)$ besitzt positive L\"osungen (z.B. f\"ur $u(x,0)=u_0(x)$ f\"ur jedes $0\le u_0\in L^2(\mathbb R^N)$) falls $c\le C^*(N)$ und keine positive L\"osungen falls $c>C^*(N)$.
\end{thm}
\begin{proof}[Beweisidee]
Sei $V_n(x)=\inf\{V_c(x), n\}$ das \emph{cutoff}-Potential. Bezeichne $u_n$ die L\"osung von
\begin{equation}\label{cutoff}
\begin{cases}
\partial_t u_n - \Delta u_n - V_n u_n =0, t\ge 0, x\in \mathbb R^N,\\
u_n(x,0)=u(x,0)=f(x)\ge 0.
\end{cases}
\end{equation}
Dann falls $c\le C^*(N)$ k\"onnen wir zum Grenzwert $n\to \infty$ \"ubergehen und erhalten eine L\"osung.  Die Schwierigkeit liegt im Erhalt der Konvergenz.
\end{proof}

\section{Das Cabr\'e--Martel Theorem}

Wir definieren den Grundzustand von $-(\Delta +V)$ durch
\begin{equation}
\lambda_1(\Delta + V):= \inf_{\phi \in H^1(\mathbb R^N)\setminus \{0\}} \left ( \frac{\int_{\mathbb R^N} |\nabla \phi|^2\, \mathrm dx - \int_{\mathbb R^N} V\phi^2\, \mathrm dx}{\int_{\mathbb R^N} \phi^2\, \mathrm dx} \right ).
\end{equation}
X. Cabr\'e und Y. Martel zeigten 1999 den Zusammenhang zwischen Existenz von schwachen L\"osungen von \eqref{heat} und der Existenz eines Grundzustands. 
\begin{thm}[Cabr\'e--Martel Theorem]
Sei $0\le V\in L^1_{\text{loc}}(\mathbb R^N), N\ge 3$.  Es folgt:
\begin{enumerate}[(i)]
\item Falls $\lambda_1(\Delta + V) > -\infty$, dann existiert eine postive L