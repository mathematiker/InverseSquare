\documentclass[11pt,twoside,a4paper]{article}
\usepackage[a4paper, scale=0.76]{geometry}
\usepackage[utf8]{inputenc}
\usepackage[ngerman]{babel} 	% Deutsche Sprachunterst\"utzung f\"ur Silbentrennung, etc.
\usepackage[T1]{fontenc}		% Erlaubt das kopieren/suchen von/nach Umlauten im pdf-Dokument
		% Encoding der tex-Dateien, erlaubt direkte Eingabe von Sonderzeichen
\usepackage{lmodern}			% Sch\"onere Schriftart f\"ur pdf-Dokumente

\usepackage{enumerate}			% Aufz\"ahlungen wie a), (I), etc. als optionales Argument
\usepackage{parskip}			% Entfernt Einr\"uckung bei Abs\"atzen und f\"ugt vertikalen Abstand ein
\usepackage{float}              % Erlaubt [H] Option f\"ur table/figure
%\usepackage{scrhack} 			% Wegen float-package, das eine warnung mit scrbook erzeugt

\usepackage{algpseudocode}		% algorithmicx package f\"ur Pseudo-Code
\usepackage{verbatim}			% F\"ur Ascii-Art ;-)

\usepackage{amsmath}
\usepackage{amssymb}
\usepackage{amsthm}
\usepackage{mathtools}
\mathtoolsset{showonlyrefs}
\usepackage{mathdots}

\usepackage{fancyhdr} % Required for custom headers
\usepackage{lastpage} % Required to determine the last page for the footer
\usepackage{extramarks} % Required for headers and footers
\usepackage{graphicx} % Required to insert images
%\usepackage{lipsum} % Used for inserting dummy 'Lorem ipsum' text into the template
\usepackage{amsmath}
%\usepackage{amsfont}
%\usepackage{amssymb}

\newtheorem{thm}{Theorem}
\newtheorem{st}[thm]{Satz}
\newtheorem{lem}[thm]{Lemma}
\newtheorem{df}[thm]{Definition}
\newtheorem{kor}[thm]{Korollar}
\newtheorem{prop}[thm]{Proposition}
\newtheorem{alg}[thm]{Algorithmus}

%\theoremheadertypefont{\color{black}} % Font f\"ur Theorem-Typ (Satz, Definition, etc.)
\theoremstyle{break}

\newtheorem{nt}[thm]{Bemerkung}
\newtheorem{ex}[thm]{Beispiel}

\newtheorem*{note}{Bemerkung}

%\usepackage{multicol}
% Margins
%\topmargin=-2in
%\evensidemargin=1in
%\oddsidemargin=1in
%\textwidth=7.5in
%\textheight=9.0in
%\headsep=0.5in 

\pagestyle{fancy}

\rhead{Matthias \textsc{Hofmann}} % Top right header
\lhead{Schr\"odinger mit inversquadratischem Potential\\
\today}
\chead{ }
%\title{}

\begin{document}
\cleardoublepage
\begin{titlepage}

\begin{center}


% Oberer Teil der Titelseite:

\textsc{\LARGE Universit\"at Stuttgart}\\[1.5cm]

\textsc{\Large Masterseminar Funktionalanalysis}\\[0.5cm]


% Title
\newcommand{\HRule}{\rule{\linewidth}{0.5mm}}
\HRule \\[0.4cm]
{ \huge \bfseries Schr\"odingeroperatoren mit inversquadratischem Potential}\\[0.4cm]

\HRule \\[1.5cm]

% Author and supervisor

\begin{center} \Large
Matthias \textsc{Hofmann}
\end{center}

\hfill

\vfill

% Unterer Teil der Seite
{\large \today}

\end{center}

\end{titlepage}

\thispagestyle{empty}
\mbox{}\newpage
\addtocounter{page}{-2}
\section{Einf\"uhrung}
Wir betrachten im Folgenden die W\"armeleitungsgleichung
\begin{equation}\label{heat}
\begin{cases}
\partial_t u - \Delta u =0, t\ge 0, x\in \mathbb R^N,\\
u(x,0)=u_0 \ge 0.
\end{cases}
\end{equation}
Dann ist die L\"osung der W\"armeleitungsgleichung eindeutig durch
\begin{equation}
u(x,t)=\int_{\mathbb R^N} K(x-y, t) u_0(y) \, \mathrm dy,
\end{equation}
gegeben ist, wobei $K(x)=\frac{1}{(4\pi t)^{n/2}} e^{-|x|^2/4t}$. Dann ist $u(x,t)$ die klassische L\"osung des Problems \eqref{heat} 
und
\begin{equation}
T(t)\phi:=e^{\Delta t}\phi:=\int_{\mathbb R^N} K(x-y,t) \phi(y)\, \mathrm dy 
\end{equation}
definiert eine stark stetige Halbgruppe auf $L^2(\mathbb R^N)$ (vgl. \cite{engel-nagel} Beispiel II.2.12). Sie erh\"alt Positivit\"at, sodass f\"ur $\phi \ge 0$ auch $T(t) \phi \ge 0$, und ist analytisch f\"ur $t>0$.

Sei $A_V= -\Delta - V(\cdot)$ ein Schr\"odinger-Operator auf $L^2(\mathbb R^N)$, sodass
\begin{equation}
0\le V\in L^\infty(\{x:|x|\ge \varepsilon\})
\end{equation}\label{perturbed}
f\"ur jedes $\varepsilon>0$. Wie verhalten sich L\"osungen der gest\"orten W\"armeleitungsgleichung 
%\begin{equation}
%\partial_t u +A_V u =0\quad (x\in \mathbb R^N, t\ge 0).
%\end{equation}
\begin{equation}\label{perturbed}
\begin{cases}
\partial_t u - A_V u =0, t\ge 0, x\in \mathbb R^N,\\
u(x,0)=u_0 \ge 0.
\end{cases}
\end{equation}
F\"ur $V\in L^\infty(\mathbb R^N)$ definiert $V$ einen beschr\"ankten Multiplikationsoperator und die Existenz von klassischen L\"osungen folgt aus Satz \ref{bounded} (s. Appendix) f\"ur beschr\"ankte St\"orungen. %\vspace{.25cm}
Was passiert, wenn $V$ zu singul\"ar wird?

Wir erlauben im Folgenden auch schwache L\"osungen von \eqref{perturbed}. \vspace{.25cm}
\begin{df}
$u$ ist schwache L\"osung von \eqref{perturbed} falls, f\"ur jedes $T, R>0$, gilt
\begin{gather}\label{weak}
u\in L^1(B(0,R) \times (0,T)), Vu \in L^1(B(0,R)\times (0,T)) \text{ und }\\
\int_0^T \int_{\mathbb R^N} u (-\partial_t \phi - L\phi) \, \mathrm dx \, \mathrm dt - \int_{\mathbb R^N} f \phi(\cdot, 0) \, \mathrm dx = \int_0^T \int_{\mathbb R^N} V u \phi \, \mathrm dx \, \mathrm dt
\end{gather}
f\"ur alle $\phi \in C_c^2(\mathbb R^N\times [0,T))$.
\end{df}

H. Brezis und J. L. Lions vermuteten, dass f\"ur $V(x) \le \frac{C}{|x|^{2-\varepsilon}}$ mit $C, \varepsilon>0$ keine positive L\"osungen besitzt. P. Baras und A. Goldstein l\"osten die Problemstellung.  Sei dazu $C^*(N)=( \tfrac{N-2}{N} )^2$ und $V_c(x) = \frac{c}{|x|^2}$. \vspace{.25cm}

\begin{thm}[\textsc{Baras--Goldstein} \textbf{1984}]\label{main}
Die Gleichung  $\partial_t u + A_{V_c} u =0, (x\in \mathbb R^N, t\ge0)$ besitzt positive L\"osungen (z.B. f\"ur $u(x,0)=u_0(x)$ f\"ur jedes $0\le u_0\in L^2(\mathbb R^N)$) falls $c\le C^*(N)$ und keine positive L\"osungen falls $c>C^*(N)$.
\end{thm}
\begin{proof}[Beweisidee]
Sei $V_n(x)=\inf\{V_c(x), n\}$ das \emph{cutoff}-Potential. Bezeichne $u_n$ die L\"osung von
\begin{equation}\label{cutoff}
\begin{cases}
\partial_t u_n - \Delta u_n - V_n u_n =0, t\ge 0, x\in \mathbb R^N,\\
u_n(x,0)=u(x,0)=f(x)\ge 0.
\end{cases}
\end{equation}
Dann falls $c\le C^*(N)$ k\"onnen wir zum Grenzwert $n\to \infty$ \"ubergehen und erhalten eine L\"osung.  Die Schwierigkeit liegt im Erhalt der Konvergenz.
\end{proof}

\section{Das Cabr\'e--Martel Theorem}

Wir definieren den Grundzustand von $-(\Delta +V)$ durch
\begin{equation}\label{groundstate}
\lambda_1(\Delta + V):= \inf_{\phi \in H^1(\mathbb R^N)\setminus \{0\}} \left ( \frac{\int_{\mathbb R^N} |\nabla \phi|^2\, \mathrm dx - \int_{\mathbb R^N} V\phi^2\, \mathrm dx}{\int_{\mathbb R^N} \phi^2\, \mathrm dx} \right ).
\end{equation}
X. Cabr\'e und Y. Martel zeigten den Zusammenhang zwischen Existenz von schwachen L\"osungen von \eqref{heat} und der Existenz eines Grundzustands. 

\vspace{.25cm}

\begin{thm}[\textsc{Cabr\'e--Martel} \textbf{1999}]\label{martel}
Sei $0\le V\in L^1_{\text{loc}}(\mathbb R^N), N\ge 3$.  Es folgt:
\begin{enumerate}[(i)]
\item Falls $\lambda_1(\Delta + V) > -\infty$, dann existiert eine postive L\"osung $u\in C([0,\infty), L^2(\mathbb R^N))$, sodass
\begin{equation}\label{exponential}
\|u(t)\|_{L^2(\mathbb R^N)} \le e^{\omega t} \|u_0\|_{L^2(\mathbb R^N)}, t\ge 0
\end{equation}
f\"ur ein $\omega\in \mathbb R$.
\item Falls $\lambda_1(\Delta + V)=-\infty$, dann folgt f\"ur $0\le u_0 \in L^2(\mathbb R^N)\setminus\{0\}$, dann gibt es keine positive L\"osung von \eqref{heat}, sodass \eqref{exponential} erfüllt wird.
\end{enumerate}
\end{thm}
\begin{proof}
%\begin{enumerate}[(i)]
(i) Betrachte im Folgenden die L\"osungen $u_n$ von
\begin{equation}\label{cutoff2}
\begin{cases}
\partial_t u_n - \Delta u_n - V_n u_n =0, t\ge 0, x\in \mathbb R^N,\\
u_n(x,0)=u(x,0)=f(x)\ge 0.
\end{cases}
\end{equation}
Nach Theorem \ref{bounded} erzeugt $\Delta + V_n$ eine positive, analytische, stark stetige Halbgruppe $S_n(\cdot)$ auf $L^2(\mathbb R^N)$. Es folgt mit \eqref{duhamel} (vgl. Appendix)
\begin{equation}\label{beh1}
0 \le u_n \le u_{n+1}, \quad n=1,2,3,\ldots
\end{equation}
Multiplizieren wir \eqref{cutoff2} mit $u_n$ und integrieren, dann folgt
\begin{equation}
\frac{1}{2} \int_{\mathbb R^N} \partial_t (u_n)^2\, \mathrm dx \le - \int_{\mathbb R^N} |\nabla u_n|^2 + \int_{\mathbb R^N} V u_n^2\, \mathrm d\mu.
\end{equation}
So erhalten wir mit der Definition von \eqref{groundstate}
\begin{equation}
\frac{1}{2} \int_{\mathbb R^N} \partial_t (u_n)^2\, \mathrm dx \le -\lambda_1(\Delta+V) \int_{\mathbb R^n} u_n^2 dx.
\end{equation}
Und somit
\begin{equation}
\|u_n(t)\|_{L^2(\mathbb R^N)} \le e^{-\lambda_1(\Delta +V) t} \|u_0\|_{L^2(\mathbb R^N)}, t\ge 0.
\end{equation}
Daher folgt die lokal gleichm\"a{\ss}ige Beschr\"anktheit der zugeh\"origen Halbgruppen mit
\begin{equation}
\|S_n(t)\| \le e^{-\lambda_1(\Delta + V)}t
\end{equation}
Nach dem Trotter--Neveu--Kato Theorem (vgl. Appendix) folgt die Existenz einer stark stetigen Halbgruppe $S(t)$, sodass $S_n(t) \to S(t), t\ge 0$ im staken Sinne konvergiert.  Dann folgt im Grenzwert f\"ur $u(t) = S(t) u_0\in C([0,T],L^2(\mathbb R^N))$, dass dies eine schwache L\"osung zu \eqref{perturbed} ist.  

(ii) Angenommen es g\"abe eine postive schwache L\"osung zu \eqref{perturbed} mit Anfangswert $u_0\ge 0$.  Sei $u_n$ die eindeutige positive L\"osung zu \eqref{cutoff2}. Dann gilt das Maximumsprinzip (vgl. Appendix)
\begin{equation}\label{maximum}
0<u_n \le u.
\end{equation}
Mit monotoner Konvergenz folgt die Existenz eines punktweisen Grenzwerts $u_n(t)\to \tilde u(t), t\ge 0$, die sogenannte milde L\"osung des Problems.  

Multiplizieren wir \eqref{cutoff2} mit $\frac{\phi^2}{u_n}$ und integrieren wir, so ergibt sich
\begin{equation}
\int_{\mathbb R^N} V_n \phi^2\, \mathrm dx \le \partial_t \left ( \int_{\mathbb R^N} (\log u_n) \phi^2\, \mathrm dx \right )+ \int_{\mathrm R^N} |\nabla \phi|^2\, \mathrm dx.
\end{equation}
Integrieren wir f\"ur $t\in (1,\infty)$, so folgt
\begin{equation}
(t-1) \int_{\mathbb R^N} V_n \phi^2\, \mathrm dx \le \int_{\mathbb R^N} \log \left ( \frac{u_n}{u_0} \right )\phi^2\, \mathrm dx + (t-1) \int_{\mathbb R^N}|\nabla \phi|^2\, \mathrm dx
\end{equation}
für jedes $t>1$. Lassn wir $n\to \infty$ folgt mit Lebesgue
\begin{equation}
\int_{\mathbb R^N} V\phi^2 \, \mathrm d\mu - \int_{\mathbb R^N} |\nabla \phi|^2 \, \mathrm dx \le \frac{1}{t-1} \left [ \int_{\mathbb R^N} \log(\tilde u(t)) \phi^2 \, \mathrm dx -\int_{\mathbb R^N} \log(\tilde u(1))\phi^2\, \mathrm dx \right ]
\end{equation}
für jedes $t>1$. Mit Jensenscher und Hölderscher Ungleichungen folgt
\begin{align*}
\int_{\mathbb R^N} V\phi^2 \, \mathrm dx &- \int_{\mathbb R^N} |\nabla \phi|^2\, \mathrm dx \\
 &\le  \frac{1}{2(t-1)} \left \{ 2\log(M) + 2\omega t + 2\log\|u_0\|_{L^2(\mathbb R^N)} + 2 \log \|\phi\|_\infty - 2\int_{\mathbb R^N} \log(\tilde u(1))\phi^2\, \mathrm dx \right \}.
\end{align*}
Im Grenzwert $t\to \infty$, erhalten wir
\begin{equation}
\int_{\mathbb R^N} V\phi^2\, \mathrm dx - \int_{\mathbb R^N} |\nabla \phi|^2\, \mathrm dx \le \omega \int_{\mathbb R^N} \phi^2\, \mathrm dx.
\end{equation}
Mit Dichtheit von $C_c^\infty(\mathbb R^N)$ in $H_0^1(\mathbb R^N)$ folgt $\lambda_1(\Delta +V) >-\infty$. Ein Widerspruch zur Voraussetzung.
%\end{enumerate}
\end{proof}
\begin{nt}\label{bem}
Die Bedingung \eqref{exponential} in der Nichtexistenz ist tatsächlich nicht notwendig. Dies hängt mit dem Maximumsprinzip \eqref{maximum} zusammen. In \cite{baras-goldstein} wurde mittels eines Blowup-Arguments gezeigt, dass die Lösungen $u_n(x,t)$ für $t>0$ überall und zu allen Zeiten unbeschränkt sind.
\end{nt}
\section{Hardy-Ungleichung und das Baras--Goldstein-Theorem}
Mittels Theorem \ref{martel} in Verbindung mit Bemerkung \ref{bem} können wir Theorem  \ref{main} beweisen. Tatsächlich hängt die Existenz von Lösungen stark mit der Hardy-Ungleichung und Optimalität der Konstanten in dieser zusammen.
\begin{lem}[Hardy-Ungleichung]
Sei $N\ge 3$. Dann gilt
\begin{equation}\label{hardy}
C^*(N) \int_{\mathbb R^N} \frac{\phi^2(x)}{|x|^2}\, \mathrm dx \le \int_{\mathbb R^N} |\nabla \phi|^2\, \mathrm dx, \quad \phi \in H^1(\mathbb R^N).
\end{equation}
\end{lem}
\begin{proof}
Es genügt die Hardy-Ungleichung für $\phi\in C_c^\infty(\mathbb R^N)$ zu zeigen. Die allgemeine Hardy-Ungleichung folgt durch ein Dichtheitsargument.
Tatsächlich finden wir eine Folge $\phi_n\to \phi$ in $H^1$ mit $\phi_n \in C_c^\infty(\mathbb R^N)$, die fast überall punktweise konvergiert. Dann folgt mit dem Lemma von Fatou die Ungleichung in allgemeiner Form.

Sei $\phi \in C_c^\infty(\mathbb R^N)$. Es folgt
\begin{equation}
\phi(x)=-\int_1^\infty \frac{\mathrm d}{\mathrm dt} \phi(tx) \, \mathrm dt= - \int_1^\infty x \cdot (\nabla \phi(tx))\, \mathrm dt.
\end{equation}
Mit der Miskowskischen Integralungleichung folgt
\begin{equation}
\left \| \frac{\phi}{|x|} \right \|_{L^2(\mathbb R^N)} \le \int_1^\infty \left \| \nabla \phi(t\cdot) \right \|_{L^2(\mathbb R^N} \, \mathrm dt = \left ( \int_1^\infty t^{-N/2} \, \mathrm dt \right ) \|\nabla \phi\|_{L^2(\mathbb R^N)}=\sqrt{C^*(N)} \|\nabla\phi\|_{L^2(\mathbb R^N)}.
\end{equation}
{\it Alternativ:} Für $x\in \mathbb R^N\setminus\{0\}$ folgt
\begin{equation}
\operatorname{div} \left ( \frac{\lambda x}{|x|^2} \right ) = \frac{\lambda (N-2)}{|x|^2}
\end{equation}
wobei $\lambda>0$ noch frei wählbar sei. 

Sei $\phi \in C_c^\infty(\mathbb R^N\setminus \{0\})$. Multiplizieren wir beide Seiten mit $\phi^2$ und integrieren wir partiell erhalten wir 
\begin{equation}
-2\int_{\mathbb R^N} \phi \frac{\lambda x}{|x|^2} \cdot \nabla \phi \, \mathrm dx = \int_{\mathbb R^N} \frac{\phi^2\lambda (N-2)}{|x|^2}\, \mathrm dx.
\end{equation}
Mit der Youngschen Ungleichung folgt dann
\begin{equation}
\int_{\mathbb R^N} \frac{\phi^2\lambda (N-2)}{|x|^2} \, \mathrm dx \le 2 \int_{\mathbb R^N} \left | \frac{\lambda \phi}{|x|} \right | |\nabla \phi(x)|\, \mathrm dx \le \int_{\mathbb R^N} \left ( \frac{\lambda \phi}{|x|} \right )^2\, \mathrm dx + \int_{\mathbb R^N} |\nabla \phi|^2\, \mathrm dx.
\end{equation}
Es folgt somit
\begin{equation}
(\lambda(N-2) -\lambda^2)\int_{\mathbb R^N} \frac{\phi^2}{|x|^2}\, \mathrm dx \le \int_{\mathbb R^N} |\nabla \phi|^2\, \mathrm dx.
\end{equation}
$\lambda\mapsto \lambda(N-2)-\lambda^2$ nimmt sein Maximum bei $\lambda= \frac{N-2}{2}$ an und es ergibt sich \eqref{hardy}. Die Aussage überträgt sich mit Dichtheit wegen $H^1(\mathbb R^N) = H_0^1(\mathbb R^N\setminus\{0\})$.
\end{proof}

\emph{Alternative 2:} Wie bereits erwähnt genügt es die Gleichung für $\phi \in C_c^\infty(\mathbb R^N\setminus\{0\})$ zu zeigen.  Sei $\mathcal R$ das radiale Vektorfeld $\mathcal R=\sum_{i=1}^d x_i \partial_{i}$. Da $\mathcal R|x|^{-2}=-2|x|^{-2}$ führt partielle Integration zu
\begin{equation}
\int_{\mathbb R^N} \frac{|f(x)|^2}{|x|^2}\, \mathrm dx = \frac{1}{2} \int_{\mathbb R^d} \frac{2f(x) \mathcal Rf(x)}{|x|^2} \, \mathrm dx + \frac{d}{2} \int_{\mathbb R^d} \frac{|f(x)|^2}{|x|^2} \, \mathrm dx.
\end{equation}
Mit der Cauchy-Schwarz-Ungleichung folgt
\begin{align*}
\int_{\mathbb R^N} \frac{|f(x)|^2}{|x|^2} \, \mathrm dx &=  \frac{2}{2-d} \int_{\mathbb R^d} \frac{f(x) \mathcal Rf(x)}{|x|^2}\, \mathrm dx\\
&\le \frac{2}{d-1}\left ( \int_{\mathbb R^N} \frac{|f(x)|^2}{|x|^2}\, \mathrm dx \right )^{1/2} \left ( \int_{\mathbb R^N} \frac{|\mathcal R f(x)|^2}{|x|^2} \, \mathrm dx \right )^{1/2},
\end{align*}
und damit
\begin{equation}
\left (\int_{\mathbb R^N} \frac{|f(x)|^2}{|x|^2} \right )^{1/2}\le \frac{2}{d-2} \left ( \int_{\mathbb R^N} |\nabla f(x)|^2\, \mathrm dx \right )^{1/2}.
\end{equation}

\begin{proof}[Beweis von Theorem \ref{main}]
Die Existenz von Lösungen folgt für $c\le C^*(N)$ direkt aus Theorem \ref{martel}, da mit der Hardy-Ungleichung \eqref{hardy} folgt
\begin{equation}
\lambda_1(\Delta + \frac{c}{|x|^2})= \inf_{0\neq \phi \in H^1(\mathbb R^N)} \left ( \frac{\int_{\mathbb R^N} |\nabla \phi|^2 \, \mathrm dx - \int_{\mathbb R^N} c \frac{\phi^2}{|x|^2}}{\phi^2} \right )\ge 0.
\end{equation}
Sei $c>C^*(N)$. Um die Nichtexistenz von Lösungen zu zeigen, wählen wir für $\frac{2-N}{2}<\gamma <0$ geeignete $\phi_\gamma\in H^1(\mathbb R^N)$, sodass
\begin{equation}
\lambda_1(\Delta + \frac{c}{|x|^2})\le \lim_{\gamma \to (\tfrac{2-N}{N})+} \left ( \frac{\int_{\mathbb R^N} |\nabla \phi_\gamma|^2\, \mathrm dx- \int_{\mathbb R^N} c \frac{\phi^2_{\gamma}}{|x|^2}\, \mathrm dx}{\int_{\mathbb R^N} \phi^2_\gamma  \, \mathrm dx} \right )=-\infty.
\end{equation}
Die Aussage folgt dann aus Theorem \ref{martel}. Dies zeigt insbesondere die Optimalität der Konstanten $C^*(N)$ in \eqref{hardy}.

Sei also $\phi_\gamma= |x|^\gamma \eta(x)$, wobei $x\in \mathbb R^N$ und $\eta\in C_c^\infty(\mathbb R^N)$, sodass $\eta =1$ auf $B_1(0)$ und $\eta=0$ auf $\mathbb R^N\setminus B_2(0)$.  Es folgt
\begin{align*}
\int_{\mathbb R^N} \phi^2_\gamma(x)\, \mathrm dx &= \int_{B_1} |x|^{2\gamma} \, \mathrm dx + \int_{B_2\setminus B_1} |x|^{2\gamma} \eta^2\, \mathrm dx\\
&\ge \frac{\sigma_{N-1}}{2\gamma+N} + 2^{2\gamma} c_1,\\
\int_{\mathbb R^N} |\nabla \phi_\gamma|^2(x)\, \mathrm dx &\le \frac{\gamma^2 \sigma_{N-1}}{2\gamma +N-2}+ 2(\gamma^2 c_1+ \| \nabla \eta\|^2_\infty),\\
\int_{\mathbb R^N} \frac{c\phi_\gamma^2(x)}{|x|^2} \, \mathrm dx &\ge \frac{c\sigma_{N-1}}{2\gamma+N-2} + c 2^{\gamma -2}c_1,
\end{align*}
wobei $\sigma_{N-1}$ der Flächeninhalt der Einheitskugel im $\mathbb R^N$ ist und $c_1=\int_{B_2 \setminus B_1} \eta^2(x)\, \mathrm dx$. Damit folgt
\begin{equation}
\lambda_1(\Delta + \frac{c}{|x|^2}) \le \lim_{\gamma \to (\tfrac{2-N}{N})+} \left ( \frac{\int_{\mathbb R^N} |\nabla \phi_\gamma|^2\, \mathrm dx- \int_{\mathbb R^N} c \frac{\phi^2_{\gamma}}{|x|^2}\, \mathrm dx}{\int_{\mathbb R^N} \phi^2_\gamma  \, \mathrm dx} \right )=-\infty,
\end{equation}
was den Beweis abschließt.


\end{proof}
\begin{nt}
Theorem \ref{martel} wurde in \cite{goldstein-rhandi} für gestörte Kolmogorov-Gleichungen
\begin{equation}%\label{perturbed}
\begin{cases}
\partial_t u - (\Delta+V) u +\nabla \rho \cdot \nabla u=0, t\ge 0, x\in \mathbb R^N,\\
u(x,0)=u_0 \ge 0
\end{cases}
\end{equation}
studiert (für eine geeignete Einführung verweisen wir auf \cite{lorenzi}) und für $\rho(x)=\frac{1}{2} Bx\cdot x$, wobei $B\in \mathbb R^{N\times N}$ positive definite Matrix, beschreibt dies gerade einen \textsc{Ornstein--Uhlenbeck}-Prozess.  Für solche wurde in \cite{goldstein-rhandi} ein Analogon zum Baras--Goldstein-Theorem zur Existenz von Lösung durch den Beweis einer gewichteten Hardy-Ungleichung gezeigt.
\end{nt}
\appendix
\section{Appendix}
In diesem Teil geben wir einige relevante Aussagen mit Referenz an.  Wir beginnen mit einem elementaren Satz aus der Störungstheorie:

\vspace{.25cm}

\begin{thm}\label{bounded}
Sei $(A, D(A))$ erzeugt von einer stark stetigen Halbgruppe $(T(t))_{t\ge 0}$ auf einem Banachraum $X$, sodass
\begin{equation}
\|T(t)\|\le M e^{\omega t}
\end{equation}
f\"ur alle $t\ge 0$ und $w\in \mathbb R$, $M\ge 1$. Falls $B\in \mathcal L(X)$, dann erzeugt $C:=A+B$ mit $D(C)=D(A)$ eine stark stetige Halbgruppe $(S(t))_{t\ge 0}$ mit
\begin{equation}
\|S(t)\|\le M e^{(w+ M\|B\|) t}
\end{equation}
f\"ur alle $t\ge 0$. Diese erf\"ullt das \emph{\textsc{Duhamel}-Prinzip}:
\begin{equation}\label{duhamel}
S(t) x = T(t) x + \int_0^t T(t-s) B S(s) x \,\mathrm ds
\end{equation}
f\"ur alle $t\ge 0$ und $x\in X$. 
Außerdem lässt sich diese schreiben durch die stark konvergente \textsc{Dyson--Phillips}-Reihe
\begin{equation}
S(t)=\sum_{n=0}^\infty \tilde S_n(t),
\end{equation}
wobei $\tilde S_0(t):=T(t)$ und
\begin{equation}
\tilde S_{n+1} := \int_0^t T(t-s) B\tilde S_n(s)\, \mathrm dx.
\end{equation}

Insbesondere \"ubertragen sich Eigenschaften wie Postivit\"at (falls $B\ge 0$) und Analytizit\"at.
\end{thm}
\begin{proof}
Wir verweisen auf \cite{engel-nagel} Satz III.1.3, Korollar III.1.7 und Satz III.1.10. 
%Die Positivit\"at der Halbgruppe folgt aus \cite{engel-nagel} Korollar VI.2.4 und die Analytiz\"at nach Theorem III.2.10. 
\end{proof}

\begin{lem} Seien $u_n$ Lösungen von \eqref{cutoff2}, dann gilt
\begin{equation}
0 <u_n \le u_{n+1}.
\end{equation}
\end{lem}
%Sei wie vorher seien $u_n$ die Lösungen von \eqref{cutoff2}. Dann folgt i
\begin{proof}
In direkter Konsequenz zu vorigem Theorem folgt $u_n\ge e^{\Delta t} u_0> 0$ für $t>0$. Dann folgt mit \eqref{duhamel} 
\begin{equation}
u_{n+1}(t) -u_n(t)= \int_0^t e^{\Delta + V_n} (V_{n+1}-V_n) u_{n}(s)\, \mathrm ds\ge 0.
\end{equation}
\end{proof}
%Also folgt:

Als nächstes wollen wir das Maximumsprinzip zeigen:
\begin{lem} Seien $u_n$ Lösungen von \eqref{cutoff2} und positive Lösung von \eqref{perturbed}, dann gilt
\begin{equation}
0<u_n(t)\le u(t), \quad t\ge 0.
\end{equation}
\end{lem}
\begin{proof}
Mit der Definition von schwachen Lösungen folgt
\begin{equation}
\int_0^T \int_{\mathbb R^N} (u_n-u) (-\partial_t \phi - \Delta \phi - V_n \phi) \, \mathrm dx \, \mathrm dt = \int_0^T \int_{\mathbb R^N} (V_n -V) u\phi \, \mathrm dx \, \mathrm dt \le 0
\end{equation} 
für alle $0\le \phi \in H_0^2(B_R \times [0, T))$ und $T,R>0$.  Sei $0 \le \psi \in C_c^\infty(B_R\times (0,T))$, so können wir $0 \le  \phi$ derart wählen, dass 
\begin{equation}
\begin{cases}
\partial_t \phi + \Delta \phi + V_n \phi = - \psi.\\
\phi\big|_{\partial B_R \times (0,T)} =0,\\
\phi(\cdot ,T)=0, \quad \phi(\cdot, t)
\end{cases}
\end{equation}
Wir können 
Diese ist äquivalent zur Existenz von $\tilde \phi(x,t)=\phi(x,T-t)$ mit
\begin{equation}
\begin{cases}
\partial_t \tilde \phi(x,t) = \Delta \tilde \phi(x,t)+V_n \tilde \phi(x,t)+\psi(x,t),\\
\tilde \phi\big|_{\partial B_R \times (0,T)} =0,\\
\tilde \phi(\cdot,0)=0
\end{cases}
\end{equation}
die aus der Existenz von Lösungen von diesen Problemen zurückzuführen sind. Dafür kann man analog wie in \cite{arendt} Theorem 6.2.8 verfahren oder allgemeiner auf \cite{lady} Theorem 2.2.5 zurückgreifen.
und somit folgt 
\begin{equation}
0<u_n \le u, \quad t\ge 0.
\end{equation}
\end{proof}

Als Letztes wollen wir einen Beweis zu einer Variation des \textsc{Trotter--Never--Kato}-Theorems angeben.  Der Folgende mit Verallgemeinerung ist zum Beispiel für kontraktive Halbgruppen zu finden als Proposition 3.6 in \cite{arendt-goldstein}, der aber aus \cite{voigt} stammt.

\vspace{.25cm}

\begin{thm}[\textsc{Trotter--Neveu--Kato}]
Seien $T_k$ stark stetige, quasikontraktive Halbgruppe  auf $L^2$, so dass $0\le T_k(t) \le T_{k+1}(t)$ und
\begin{equation}
\|T_k(t)\|\le e^{\omega t},\quad \omega \in \mathbb R.
\end{equation}  
Dann existiert $T(t)f=\lim_{k\to \infty} T_k(t)f$ für alle $f\in L^2, t\ge 0$, und definiert eine stark stetige Halbgruppe $T$ auf $L^2$. 
\end{thm}

\begin{proof}
Die Existenz des starken Grenzwerts folgt mit monotoner Konvergenz und $T(t)\in \mathcal L(L^2)$ und $T(t+s)=T(t) T(s)$ für $t,s\ge 0$.  Es bleibt starke Stetigkeit des Grenzwerts zu zeigen.  Sei $t_n\downarrow 0$, $0\le f\in L^2$.  Dann ist zu zeigen: $f_n:=T(t_n)f\to f$ wenn $n\to \infty$. Sei $g_n:=T_1(t_n) f$. Dann $0\le g_n \le f_n$ und $g_n \to f$ wenn $n\to \infty$. Weiter folgt $\|g_n\|_{L^2} \le \|f\|_{L^2}$.

Es genügt zu zeigen, dass jede Teilfolge von $(f_n)$ eine konvergente Teilfolge besitzt.  OBdA können wir annehmen, folgt dass $f_n$ schwach gegen $h\in L^p$ konvergiert (sonst wähle eine Teilfolge).  Da $g_n \le f_n$ und $g_n \to f$ folgt, dass $f\le h$.  Damit ergibt sich $\| f\|_{L^p}\le \|h\|_{L^p}$. 
Insbesondere $g_n \to f$ in $L^2$ und wir folgern
\begin{equation}
\| h\|_{L^2(\mathbb R^n)}  \le \liminf_{n\to \infty} \| f_n\|_{L^2(\mathbb R^n}\le \liminf_{n\to \infty} e^{\omega t_n} \|f\|_{L^2(\mathbb R^n)}=\| f\|_{L^2(\mathbb R^n)}
\end{equation}
und daher $f_n\to f$ in $L^2(\mathbb R^n)$.
\end{proof}
 





\begin{thebibliography}{tief}
\bibitem{arendt} Arendt, Wolfgang; Batty, Charles J. K.; Hieber, Matthias; Neubrander, Frank:
{\it Vector-valued Laplace transforms and Cauchy problems}.
Monographs in Mathematics, 96. Birkhäuser Verlag, Basel, 2001. xii+523 pp.
\bibitem{arendt-goldstein} Arendt, Wolfgang; Goldstein, Gisèle Ruiz; Goldstein, Jerome A.: 
{\it Outgrowths of Hardy's inequality}. Recent advances in differential equations and mathematical physics, 51–68,
Contemp. Math., 412, Amer. Math. Soc., Providence, RI, 2006.
\bibitem{baras-goldstein} Baras, Pierre; Goldstein, Jerome A:
{\it The heat equation with a singular potential}.
Trans. Amer. Math. Soc. 284 (1984), no. 1, 121-139.
\bibitem{cabre-martel} Cabr\'e, Xavier; Martel, Yvan:
{\it Existence versus explosion instantan\'ee pour des \'equations de la chaleur lin\'eaires avec potentiel singulier}. (French. English, French summary) [Existence versus instantaneous blowup for linear heat equations with singular potentials]
C. R. Acad. Sci. Paris S\'er. I Math. 329 (1999), no. 11, 973-978. 
\bibitem{engel-nagel} Engel, Klaus-Jochen; Nagel, Rainer:
{\it A short course on operator semigroups}. Universitext. Springer, New York, 2006.
\bibitem{goldstein-rhandi} Goldstein, G. R.; Goldstein, J. A.; Rhandi, A.: {\it Weighted Hardy's inequality and the Kolmogorov equation perturbed by an inverse-square potential}.
Appl. Anal. 91 (2012), no. 11, 2057-2071. 
\bibitem{lady} Lady\v{z}enskaja, O. A.; Solonnikov, V. A.; Ural'ceva, N. N.
{\it Linear and quasilinear equations of parabolic type}. (Russian)
Translated from the Russian by S. Smith. Translations of Mathematical Monographs, Vol. 23 American Mathematical Society, Providence, R.I. 1968 xi+648 pp. 
\bibitem{lorenzi} Lorenzi, Luca; Bertoldi, Marcello
{\it Analytical methods for Markov semigroups}.
Pure and Applied Mathematics (Boca Raton), 283. Chapman \& Hall/CRC, Boca Raton, FL, 2007. xxxii+526 pp.
\bibitem{rhandi} Rhandi, Abdelaziz: {\it Heat and Ornstein-Uhlenbeck semigroups perturbed by an inverse-square potential}. 
                     Unpublished, 2015.
\bibitem{voigt} Voigt, Jürgen:
{\it Absorption semigroups, their generators, and Schrödinger semigroups}.
J. Funct. Anal. 67 (1986), no. 2, 167–205. 
\end{thebibliography} 
\end{document}
